\documentclass[]{report}
\usepackage[hidelinks]{hyperref}
\usepackage{xcolor}
\usepackage{graphicx}

\renewcommand{\thefigure}{\arabic{figure}}

% Title Page
\title{Vention Data Analysis Report}
\author{Kallil de Araujo Bezerra}


\begin{document}
\maketitle


\tableofcontents

\chapter{Visualization}

The data was received in a \textit{.csv} file, and it contains the following columns:

\begin{itemize}
	\item \textcolor{blue}{\textit{accountid}} - customer account id
	\item \textcolor{blue}{\textit{amount}} - amount of bought products
	\item \textcolor{blue}{\textit{closedate}} - date in which the deal was closed
	\item \textcolor{blue}{\textit{opportunityid}} - the id of a possible sales lead
	\item \textcolor{blue}{\textit{opportunity\_creation\_date\_\_c}} - date in which the first contact with a customer was made
	\item \textcolor{blue}{\textit{ownerid}} - seller id (the \textit{owner} of the opportunity)
	\item \textcolor{blue}{\textit{primary\_application\_\_c}} - primary application of the product that is being sold
	\item \textcolor{blue}{\textit{stagename}} - in which stage the sale is categorized
	\item \textcolor{blue}{\textit{sales\_team\_\_c}} - to which sales team the opportunity, seller, and customer belongs
\end{itemize}

The description of the columns are based solely on the name of the columns itself. In an ideal environment I would get a more accurate definition of each one by asking the person that sent me the file a description of each column, and proceed with the analysis to avoid mistakes and misinterpretations.

\section{Closed Won - Success sales by sales team}

Under the column \textit{stagename} it is possible to see 8 different values, they are:

\begin{itemize}
	\item Closed Lost
	\item Closed Won
	\item Prospect
	\item Project Discovery
	\item Closing Stage
	\item Project Quoted
	\item Design Review
	\item Awaiting Purchase
\end{itemize}

To create a chart that shows which sales had a positive result it was considered the registers that have the value \textit{Closed Won}.

\begin{figure}[htb]
	\centering
	\includegraphics[width=1\textwidth]{fig_01_closed_won}
	\caption{Closed Won versus Sales Team.}
	\label{fig:fig_01}
\end{figure}

As we can see in figure \textcolor{blue}{\ref{fig:fig_01}} the team with most sales is the \textit{Strategic Sales team}, with 266 closed sales won, followed by \textit{HVS 3} and \textit{HVS 1}, with 248 and 216 sales respectively. From now on the analysis will focus on them.

However, it is also important to see the success rate of each team. Even if a team wins many sales, it doesn't mean that it is the most successful because it may be losing many sales too.

To better evaluate this, it was created a \textit{Sales Index}, which can be calculated using the following formula:

\begin{equation}
	Sales\:Index = \frac{\#\:Closed\:Won}{\#\:Closed\:Won + \#\:Closed\:Lost}
\end{equation}

\chapter{SQL query}

\chapter{Technical details}

\end{document}          
